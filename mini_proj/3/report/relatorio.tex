\documentclass[a4paper]{article}

\usepackage[portuguese]{babel}
\usepackage{comment}
\usepackage[T1]{fontenc}
\usepackage[utf8]{inputenc}
\usepackage{hyperref}
\usepackage{graphicx}
\usepackage{float}
\usepackage{multirow}
\usepackage{indentfirst}
\usepackage[hypcap]{caption} % makes \ref point to top of figures and tables
\usepackage[]{algorithm2e}
\usepackage{tabularx}
\usepackage{lscape}


\begin{document}

\begin{titlepage}

	\begin{center}

		\includegraphics[width=6cm]{./title}\\[3cm]

		\textsc{\LARGE Algoritmia e Desempenho em Redes de Computadores}\\[1.5cm]

		\textsc{\Large 2º Mini Projeto - \textit{Inter-domain routing}}\\[1.5cm]


		


		\noindent
		\begin{minipage}{0.4\textwidth}
			\begin{flushleft} \large
				Bernardo Gomes, 75573
			\end{flushleft}
		\end{minipage}
		\begin{minipage}{0.4\textwidth}
			\begin{flushright} \large
				Tomás Falcato, 75876
			\end{flushright}
		\end{minipage}

		\vfill

		{\large \today}


	\end{center}

\end{titlepage}
\hypersetup{%
    pdfborder = {0 0 0}
}

\pagenumbering{arabic}
\section{Descrição do problema}
Neste mini-projeto pretende-se avaliar a conectividade de uma rede representada por um grafo, ou seja, determinar qual o número mínimo de nós que ao retirar não permitem que o grafo esteja ligado.

Numa primeira fase, dado um par de nós do grafo, calcula-se o número mínimo de nós que é necessário retirar para que não haja nenhum caminho a ligar o par especificado. 

Posteriormente, fez-se uma análise mais detalhada da rede, repetindo o processo anterior para todos os pares de nós de forma a armazenar a informação do número de nós que foi necessário retirar para cada par.  Com a informação anterior, torna-se possível calcular a probabilidade cumulativa do número de nós a separar um par de nós.

Por fim, é verificado qual o número mínimo de nós que previne o grafo de ser conexo e é mostrado ao utilizador quais são os identificadores dos mesmos, por forma a que a informação do programa possa ser facilmente comprovada.

\section{Abordagem ao problema}
Sendo do nosso conhecimento que o número mínimo de nós que separa a fonte do destino é igual ao número máximo de caminhos independentes entre a origem e o destino, é de notar que o problema em causa é bastante semelhante à determinação de quais as arestas de um grafo que previnem que este seja conexo. No caso anterior, o problema seria reduzido a um problema de fluxos. Ora recorrendo à técnica de \textit{vertex splitting} será possível resolver o problema de forma semelhante.

Ao dividir cada nó em dois, define-se $"$nó -$"$ como o nó que irá receber todas as ligações, que o nó respetivo do grafo inicial receberia, entre os restantes e ele mesmo e que apenas tem ligação ao nó com o mesmo identificador que ele ($"$nó +$"$). Define-se $"$nó +$"$ como o nó com o mesmo identificador que o nó que lhe deu origem, mantendo este as ligações entre este nó e todos os outros, que o nó respetivo do grafo inicial teria, recebendo apenas a ligação do respetivo $"$nó -$"$.

Colocando as arestas que ligam as extremidades $"$-$"$ e $"$+$"$  de cada nó com capacidade $"$1$"$ e todas as outras arestas, que já pertenciam ao grafo inicial, com capacidade infinita, aplicando o algoritmo de \textit{Ford-Fulkerson}, é-nos possível resolver o problema enunciado.

\section{Função \textit{ford\_fulkerson}}



\end{document}