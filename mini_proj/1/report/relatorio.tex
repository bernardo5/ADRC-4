\documentclass[a4paper]{article}

\usepackage[]{algorithm2e}
\usepackage[portuguese]{babel}
\usepackage{comment}
\usepackage[T1]{fontenc}
\usepackage[utf8]{inputenc}
\usepackage{hyperref}
\usepackage{graphicx}
\usepackage{float}
\usepackage{multirow}
\usepackage{indentfirst}
\usepackage[hypcap]{caption} % makes \ref point to top of figures and tables

\begin{document}

\begin{titlepage}

	\begin{center}

		\includegraphics[width=6cm]{./title}\\[3cm]

		\textsc{\LARGE Algoritmia e Desempenho em Redes de Computadores}\\[1.5cm]

		\textsc{\Large 2º Mini Projeto - \textit{Inter-domain routing}}\\[1.5cm]


		


		\noindent
		\begin{minipage}{0.4\textwidth}
			\begin{flushleft} \large
				Bernardo Gomes, 75573
			\end{flushleft}
		\end{minipage}
		\begin{minipage}{0.4\textwidth}
			\begin{flushright} \large
				Tomás Falcato, 75876
			\end{flushright}
		\end{minipage}

		\vfill

		{\large \today}


	\end{center}

\end{titlepage}
\hypersetup{%
    pdfborder = {0 0 0}
}
\pagenumbering{arabic}
\section{Descrição do problema}
Neste mini-projecto, pretende-se a implementação de diversas funções relacionadas com a transmissão de dados em pacotes consoante
o seu destino.

Nesta fase, apenas se desenvolveu funções de conversão de \textit{forwarding table} para \textit{binary tree}, desta última para
\textit{binary2-tree}, adição de novos prefixos, remoção de prefixos e impressão de uma árvore binária no formato de
\textit{forwarding table}.

\section{Função \textit{ReadTable}}
Nesta função o programa deverá ler de um ficheiro os prefixos de uma tabela e os correspondentes \textit{next hops} e criar a 
árvore binária correspondente.
Como tal, a função irá ler do ficheiro um prefixo de cada vez, adicionando-o à árvore com o \textit{next hop} correspondente.

O pseudo-código da função será o seguinte:

\begin{algorithm}[H]
 root node creation()\;
 \While{there are lines in the document}{
  get table line()\;
  
  AddPrefix()\;
 }
 \caption{ReadTable}
\end{algorithm}


\section{Considerações finais}

\end{document}